\documentclass[english]{article}

\usepackage{graphicx}
\usepackage{alltt}
\usepackage{url}
\usepackage{tabularx}
%\usepackage{ngerman}
\usepackage{longtable}
\usepackage{color}

\usepackage{xifthen}
\newboolean{showbackdoors}
\setboolean{showbackdoors}{true}  % set to false to hide subsection on backdoors for reviewing group


\newenvironment{prettytablex}[1]{\vspace{0.3cm}\noindent\tabularx{\linewidth}{@{\hspace{\parindent}}#1@{}}}{\endtabularx\vspace{0.3cm}}
%\newenvironment{prettytable}{\prettytablex{l X}}{\endprettytablex}



\title{\huge\sffamily\bfseries System Description and Risk Analysis}
\author{Cyrill Krähenbühl \and Silvan Egli \and Lukas Bischofberger}
\date{\dots}


\begin{document}
\maketitle

%% **** please observe the page limit **** 
%% (it is not allowed to change the font size or page geometry to gain more space)
%% comment or remove lines below before hand-in
\begin{center}
{\large\textcolor{red}{Page limit: 30 pages.}}
\end{center}
%%%%%%%%%%%%%%%%%%%%%%%%%%%%%%%%%%%%%%%%%%%%%%

\tableofcontents
\pagebreak


\section{System Characterization}

\subsection{System Overview}

Describe the system's mission,  the system boundaries,
and the overall system architecture, including the main subsystems and
their relationships.   This description should provide a high-level
overview of the system, e.g., suitable for managers, that complements
the more technical description that follows.


\subsection{System Functionality}

Describe the system's functions.


\subsection{Security Design}

Describe the system's security design, including key and session management and 
security of data at rest and in transit.


\subsection{Components}

List all system components and their interfaces, subdivided, for example, into
  categories such as platforms, applications, data records, etc. For
  each component, state its relevant properties.


\ifthenelse{\boolean{showbackdoors}}{
% show for handed-in version

\subsection{Backdoors}

Describe the implemented backdoors. 

\bigskip\noindent
\textbf{Hide this subsection in the version handed over to the reviewing team by setting the flag \texttt{showbackdoors} at the top of this document to \texttt{false}.}


%% do not delete the three lines below
}{ 
% empty for reviewing group's version
} 

\subsection{Additional Material}

You may have additional sections according to your needs.


\section{Risk Analysis and Security Measures}

\subsection{Assets}

Describe the relevant assets and their required security
  properties. For example, data objects, access restrictions,
  configurations, etc.

\subsection{Threat Sources}

Name and describe potential threat sources including their motivation.

\subsection{Risks Definitions}

Define likelihood, impact and risk level using the following three
  tables.

%\subsubsection{Tools}

\begin{center}
\begin{tabular}{|l|l|}
\hline
\multicolumn{2}{|c|}{\bf Likelihood} \\
\hline
Likelihood & Description \\
\hline
\hline
High   & \hspace*{20pt}\ldots \\
\hline
Medium & \hspace*{20pt}\ldots \\
\hline
Low   & \hspace*{20pt}\ldots \\
\hline
\end{tabular}
\hspace{3em}
\begin{tabular}{|l|l|}
\hline
\multicolumn{2}{|c|}{\bf Impact} \\
\hline
Impact & Description \\
\hline
\hline
High   & \hspace*{20pt}\ldots \\
\hline
Medium & \hspace*{20pt}\ldots \\
\hline
Low   & \hspace*{20pt}\ldots \\
\hline
\end{tabular}
\end{center}

\vspace{5mm}

\begin{center}
\begin{tabular}{|l|c|c|c|}
\hline
\multicolumn{4}{|c|}{{\bf Risk Level}} \\
\hline
{{\bf Likelihood}} & \multicolumn{3}{c|}{{\bf Impact}} \\ %\cline{2-4}
     & Low & Medium & High \\  \hline
 High & Low & Medium & High  \\
\hline
 Medium & Low & Medium & Medium \\
\hline
 Low & Low & Low & Low \\
\hline
\end{tabular}
\end{center}


\subsection{Risk Evaluation}

List all potential threats and the corresponding countermeasures. Estimate the risk based on the information about the threat, the threat sources and the corresponding countermeasure. Adhere to the risk definitions you have given above.


\subsubsection{{\it Evaluation Asset X}}

Evaluate the likelihood, impact and the resulting risk,  \emph{after implementation of the corresponding countermeasures}. For each threat, clearly name the threat source and the the threat action.

\begin{footnotesize}
\begin{prettytablex}{llp{5.5cm}lll}
No. & Threat &  Countermeasure(s) & L & I & Risk \\
\hline
1 & ... & ... & {\it Low} & {\it Low} & {\it Low} \\
\hline
2 & ... & ...& {\it Medium} & {\it High} & {\it Medium} \\
\hline
\end{prettytablex}
\end{footnotesize}



\subsubsection{{\it Evaluation Asset y}}

\begin{footnotesize}
\begin{prettytablex}{llp{5.5cm}lll}
No. & Threat & Countermeasure(s) & L & I & Risk \\
\hline
1 & ... & ... & {\it Low} & {\it Low} & {\it Low} \\
\hline
2 & ... & ...& {\it Medium} & {\it High} & {\it Medium} \\
\hline
\end{prettytablex}
\end{footnotesize}

\subsubsection{Detailed Description of Selected Countermeasures}

Optionally explain the details of the countermeasures mentioned above.



\subsubsection{Risk Acceptance}

List all medium and high risks, according to the evaluation above. For each risk, propose additional countermeasures that could be implemented to further reduce the risks.

\begin{footnotesize}
\begin{prettytablex}{p{2cm}X}
No. of threat & Proposed additional countermeasure including expected impact  \\
\hline
... & ... \\
\hline
... & ... \\
\hline
\end{prettytablex}
\end{footnotesize}

\end{document}

%%% Local Variables: 
%%% mode: latex
%%% TeX-master: "../../book"
%%% End: 
